\documentclass[10pt]{beamer}

\usetheme{CambridgeUS}
\usepackage[english, russian]{babel}
\usepackage[utf8]{inputenc}
\usepackage{caption}
\usepackage{etoolbox}
\usepackage{multicol}
\usepackage{listings}
\usepackage{wasysym}
\usepackage{mathtools}
\DeclarePairedDelimiter\ceil{\lceil}{\rceil}
\DeclarePairedDelimiter\floor{\lfloor}{\rfloor}

\definecolor{mygreen}{rgb}{0,0.6,0}
\lstset{
  basicstyle=\ttfamily\footnotesize,        % the size of the fonts that are used for the code
  breaklines=true,                 % automatic line breaking only at whitespace
  captionpos=b,                    % sets the caption-position to bottom
  commentstyle=\color{mygreen},    % comment style
  keywordstyle=\color{blue},       % keyword style
  stringstyle=\color{red},     % string literal style
  showstringspaces=false,
  morekeywords={include, printf},
  texcl=true     %<---- added
}


\title[\href{https://goo.gl/NRgp8K}{https://goo.gl/NRgp8K} (Term 1)]{О принципах проектирования ПО (программного обеспечения)}
\author[Гусев Илья, Булгаков Илья]{Гусев Илья, Булгаков Илья}
\institute[МФТИ] 
{Московский физико-технический институт\\*}
\date{Москва, 2018}
\subject{Computer Science}

\begin{document}

\begin{frame}
  \titlepage
\end{frame}

\begin{frame}{Содержание}
\tableofcontents
\end{frame}

\section{SOLID}
\subsection{5 принципов SOLID}

\begin{frame}[fragile]{SOLID}{5 принципов SOLID}
    \begin{itemize}
        \item \textbf{S}ingle Responsibility Principle (Принцип единственной обязанности)
        \begin{itemize}
            \item На каждый объект должна быть возложена единственная обязанность
        \end{itemize}
        \item \textbf{O}pen Сlosed Principle (Принцип открытости/закрытости)
        \begin{itemize}
            \item Программные сушности должны быть открыты для расширения, но закрыты для изменения
        \end{itemize}
        \item \textbf{L}iskov’s Substitution Principle (Принцип подстановки Барбары Лисков)
        \begin{itemize}
            \item Объекты в программе могут быть заменены наследниками без изменения свойств программы.
        \end{itemize}
        \item \textbf{I}nterface Segregation Principle (Принцип разлеления интерфейсов)
        \item \textbf{D}ependency Inversion Principle (Принцип инверсии зависимостей)
    \end{itemize}
\end{frame}

\section{YAGNI}
\subsection{«You aren't gonna need it» - «Вам это не понадобится»}
\begin{frame}[fragile]{YAGNI}{«You aren't gonna need it» - «Вам это не понадобится»}

YAGNI - процесс и принцип проектирования ПО, при котором в качестве основной цели и/или ценности декларируется отказ добавления функциональности, в которой нет непосредственной надобности.
Помогает избежать последствий:
    \begin{itemize}
        \item Тратится время, которое было бы затрачено на добавление, тестирование и улучшение необходимой функциональности.
        \item Новые функции должны быть отлажены, документированы и сопровождаться.
        \item Ненужные новые функции могут впоследствии помешать добавить новые нужные.
        \item Пока новые функции не нужны — трудно  предугадать, что они должны делать, и протестировать их.
    \end{itemize}
\end{frame}

\section{KISS}
\subsection{«Keep it simple stupid» - «Делайте вещи проще»}
\begin{frame}[fragile]{KISS}{«Keep it simple stupid» - «Делайте вещи проще»}

KISS - большинство систем работают лучше всего, если они остаются простыми.

Как применять:
    \begin{itemize}
        \item Разбивайте задачи на подзадачи. Оптимальное время решение - 4-12 часов.
        \item Каждая задача должна решаться одним или парой классов
        \item Каждый метод должен состоять не более чем из 30-40 строк. Метод должен решать одну конкретную задачу.
        \item Классы должны иметь небольшой размер.
        \item Сначала придумайте решение, потом напишите код. 
        \item Не бойтесь избавляться от кода.
    \end{itemize}
\end{frame}

\section{DRY}
\subsection{«Don’t repeat yourself» - «Не повторяйся»}
\begin{frame}[fragile]{DRY}{«Don’t repeat yourself» - «Не повторяйся»}

DRY - принцип разработки программного обеспечения, нацеленный на снижение повторения информации различного рода.
\\
Каждая часть знания должна иметь единственное, непротиворечивое и представление в рамках системы

\end{frame}

\end{document}


