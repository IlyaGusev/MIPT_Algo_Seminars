\documentclass[10pt]{beamer}

\usetheme{CambridgeUS}
\usepackage[english, russian]{babel}
\usepackage[utf8]{inputenc}
\usepackage{caption}
\usepackage{etoolbox}
\usepackage{multicol}
\usepackage{listings}
\usepackage{wasysym}
\usepackage{mathtools}
\DeclarePairedDelimiter\ceil{\lceil}{\rceil}
\DeclarePairedDelimiter\floor{\lfloor}{\rfloor}

\definecolor{mygreen}{rgb}{0,0.6,0}
\lstset{
  basicstyle=\ttfamily\footnotesize,        % the size of the fonts that are used for the code
  breaklines=true,                 % automatic line breaking only at whitespace
  captionpos=b,                    % sets the caption-position to bottom
  commentstyle=\color{mygreen},    % comment style
  keywordstyle=\color{blue},       % keyword style
  stringstyle=\color{red},     % string literal style
  showstringspaces=false,
  morekeywords={include, printf},
  texcl=true     %<---- added
}


\title[\href{https://goo.gl/NRgp8K}{https://goo.gl/NRgp8K} (Term 1)]{Реализация умных указателей}
\author[Гусев Илья, Булгаков Илья]{Гусев Илья, Булгаков Илья}
\institute[МФТИ] 
{Московский физико-технический институт\\*}
\date{Москва, 2019}
\subject{Computer Science}

\begin{document}

\begin{frame}
  \titlepage
\end{frame}

\begin{frame}{Содержание}
\tableofcontents
\end{frame}

\section{Реализация умных указателей}

\begin{frame}[fragile]{ScopedPtr}

\begin{itemize}
    \item Простой умный указатель: для хранения на стеке или в классе.
    \item Единственные владелец.
    \item Нельзя копировать и присваивать.
    \item Нельзя вернуть владение объектом.
\end{itemize}

\end{frame}

\begin{frame}[fragile]{ScopedPtr. Основные методы}

\begin{itemize}
    \item get — возвращает указатель, сохраненный внутри ScopedPtr (например, чтобы передать его в какую-то функцию)
    \item release — забирает указатель у ScopedPtr и возвращает значение этого указателя, после вызова release ScopedPtr не должен освобождать память (например, чтобы вернуть этот указатель из функции)
    \item reset — метод заставляет ScopedPtr освободить старый указатель, а вместо него захватить новый (например, чтобы переиспользовать ScopedPtr, так как оператор присваивания запрещен)
\end{itemize}

\end{frame}

\begin{frame}[fragile]{ScopedPtr. Задача}
Напишите реализацию методов для ScopedPtr
\begin{lstlisting}
struct Expression;

struct ScopedPtr
{
    // TODO
    //
    //explicit ScopedPtr(Expression *ptr = 0);
    //~ScopedPtr();
    //Expression* get() const;
    //Expression* release();
    //void reset(Expression *ptr = 0);
    //Expression& operator*() const;
    //Expression* operator->() const;
private:
    ScopedPtr(const ScopedPtr&);
    ScopedPtr& operator=(const ScopedPtr&);

    Expression *ptr_;
};
\end{lstlisting}
\end{frame}

\begin{frame}[fragile]{ScopedPtr. Реализация}

\begin{lstlisting}
struct Expression;

struct ScopedPtr
{
    explicit ScopedPtr(Expression *ptr = 0) { ptr_ = ptr; }
    ~ScopedPtr() { delete ptr_; }
    Expression* get() const { return ptr_; }
    Expression* release() { Expression *temp = ptr_; ptr_ = 0; return temp; }
    void reset(Expression *ptr = 0) { if(ptr == ptr_) { return; } else { delete ptr_; ptr_ = ptr; } }
    Expression& operator*() const { return *ptr_; }
    Expression* operator->() const { return ptr_; }
private:
    ScopedPtr(const ScopedPtr&);
    ScopedPtr& operator=(const ScopedPtr&);

    Expression *ptr_;
};
\end{lstlisting}
\end{frame}

\begin{frame}[fragile]{SharedPtr}

Умный указатель с подсчетом ссылок 

\begin{lstlisting}
{
    SharedPtr sp1( new Expression() );
    SharedPtr sp2(sp1);
}
\end{lstlisting}

Особенности:

\begin{itemize}
    \item Для разделяемых объектов.
    \item Ведётся подсчёт ссылок.
    \item Нельзя вернуть владение объектом.
\end{itemize}

\end{frame}

\begin{frame}[fragile]{SharedPtr. Задание}

Напишите реализацию методов для SharedPtr

\begin{lstlisting}
struct Expression;
struct Number;
struct BinaryOperation;

struct SharedPtr
{
    // TODO
    //
    // explicit SharedPtr(Expression *ptr = 0)
    // ~SharedPtr()
    // SharedPtr(const SharedPtr &)
    // SharedPtr& operator=(const SharedPtr &)
    // Expression* get() const
    // void reset(Expression *ptr = 0)
    // Expression& operator*() const
    // Expression* operator->() const
};
\end{lstlisting}
\end{frame}

\end{document}


