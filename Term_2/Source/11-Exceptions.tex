\documentclass[10pt]{beamer}

\usetheme{CambridgeUS}
\usepackage[english, russian]{babel}
\usepackage[utf8]{inputenc}
\usepackage{caption}
\usepackage{etoolbox}
\usepackage{multicol}
\usepackage{listings}
\usepackage{color}

\definecolor{mygreen}{rgb}{0,0.6,0}
\lstset{
  basicstyle=\ttfamily\footnotesize,        % the size of the fonts that are used for the code
  breaklines=true,                 % automatic line breaking only at whitespace
  captionpos=b,                    % sets the caption-position to bottom
  commentstyle=\color{mygreen},    % comment style
  keywordstyle=\color{blue},       % keyword style
  stringstyle=\color{red},     % string literal style
  showstringspaces=false,
  morekeywords={include, printf},
  texcl=true     %<---- added
}

\title[\href{https://goo.gl/NRgp8K}{https://goo.gl/NRgp8K} (Term 3)]{Исключения}
\author[Гусев Илья]{Гусев Илья}
\institute[МФТИ] 
{Московский физико-технический институт\\*}
\date{Москва, 2018}
\subject{Computer Science}

\begin{document}

\begin{frame}
  \titlepage
\end{frame}

\begin{frame}{Содержание}
\tableofcontents
\end{frame}

\section{Способы обработки ошибок}
\begin{frame}[fragile]{Источники ошибок}
\begin{enumerate}
    \item Внешние источники - сбои в операционной системе, сбои в железе, ошибка соединения клиентской части с серверной.
    \item Нехватка ресурсов: нехватка оперативной памяти, места на диске, ограничения на количество объектов.
    \item Ошибки в логике программы.
    \item Ошибки пользоавателя, например - неправильный ввод .
    \item Классификация условная - куда отнести деление на 0?
\end{enumerate}
\end{frame}

\begin{frame}[fragile]{Способы обработки ошибок}
\begin{enumerate}
    \item Коды возврата — если функция возвращает true — все хорошо, если же false (или любое другое специальное значение) — то произошла ошибка. Иногда есть глобальная переменная с кодом ошибки (GetLastError(), errno).
    \item Assert/abort — роняем программу или подпрограмму.
    \item Исключения — прерывистый поток исполнения.
\end{enumerate}
\end{frame}

\subsection{Коды возврата}
\begin{frame}[fragile]{Коды возврата}
\begin{lstlisting}[language=C++]
int f1()
{
  // ...
  int rc = f2();
  if (rc == 0) {
    // ...
  } else {
    // код, обрабатывающий ошибку
  }
}
\end{lstlisting}
\end{frame}

\subsection{assert}
\begin{frame}[fragile]{assert}
\begin{lstlisting}[language=C++]
#include <cassert>

int f1()
{
  // ...
  int rc = f2();
  assert(rc == 0);
  // если rc != 0, сюда мы не попадём
}
\end{lstlisting}
\end{frame}

\subsection{Исключения}
\begin{frame}[fragile]{Исключения}
\begin{lstlisting}[language=C++]
void f1()
{
  try {
    // ...
    f2();
    // ...
  } catch (some_exception& e) {
    // код, обрабатывающий ошибку
  }
}
\end{lstlisting}
\end{frame}

\section{Исключения}
\subsection{Синтаксис в C++}
\begin{frame}[fragile]{Синтаксис в C++}
\begin{lstlisting}[language=C++]
// Кинуть исключение:
throw 123;
throw 'c';
throw std::logic_error("Ошибка");

// Поймать исключение:
try {
  // Код или вызовы функций, кидающие исключение
} catch (int e) { // Вместо int - тип брошенного исключения
  // Обработка ошибки
}
\end{lstlisting}
\end{frame}

\begin{frame}[fragile]{Исключения}
\begin{enumerate}
    \item Исключение принудительно вызывает код, который распознаёт ошибку и обрабатывает его. Необработанные исключения остановливают выполнение программы.
    \item Исключение переходит к точке в стеке вызовов, в которой можно обработать ошибку. Промежуточные функций пробрасывают исключение. Им не нужно общаться с другими функциями в стеке.
    \item Механизм очистки стека исключений уничтожает все объекты в области видимости в соответствии с чётко определёнными правилами, после того, как исключение брошено.
    \item Исключение обеспечивает четкое разделение между кодом, который обнаруживает ошибку и кодом, который обрабатывает ошибку.
\end{enumerate}
\end{frame}

\subsection{Коды возврата vs исключения}
\begin{frame}[fragile]{Коды возврата vs исключения}
\begin{enumerate}
\item Исключения не накладывают на программиста обязанности проверять каждый вызов каждой функции, чтобы не потерять случайно возникшую ошибку.
\item Если ошибка не обработана на текущем уровне — ничего страшного, ей займется более верхний. Возможно, ему уже будет известно, что с ней делать.
\item Поддерживаются самим языком — нам не нужно модифицировать существующий код, чтобы научить его пробрасывать ошибку наверх этим новым способом.
\end{enumerate}
\end{frame}

\begin{frame}[fragile]{Минусы исключений}
\begin{enumerate}
\item Исключения дают рваный поток выполнения. По коду нельзя сказать куда улетит брошенное нами исключение и что может вывалится из того, что мы вызвали.
\item Реальный путь исключения по стеку определяется не статическим описанием программы (сигнатуры методов с исключениями) а её динамической структурой (кто какие реализации вызывает).
\item Исключения менее эффективны.
\item При использовании исключений нужно очень аккуратно обращаться с динамической памятью, в идеале использовать Resource Acquisition Is Initialization (RAII).
\end{enumerate}
\end{frame}

\begin{frame}[fragile]{Коды возврата vs исключения}
\begin{lstlisting}[language=C++]
void f(Number x, Number y)
{
  try {
    // ...
    Number sum  = x + y;
    Number diff = x - y;
    Number prod = x * y;
    Number quot = x / y;
    // ...
  }
  catch (Number::Overflow& exception) {
    // ...code that handles overflow...
  }
  catch (Number::Underflow& exception) {
    // ...code that handles underflow...
  }
  catch (Number::DivideByZero& exception) {
    // ...code that handles divide-by-zero...
  }
}
\end{lstlisting}
\end{frame}


\begin{frame}[fragile]{Коды возврата vs исключения}
\begin{lstlisting}[language=C++]
int f(Number x, Number y)
{
  // ...
  Number::ReturnCode rc;
  Number sum = x.add(y, rc);
  if (rc == Number::Overflow) {
    // ...code that handles overflow...
    return -1;
  } else if (rc == Number::Underflow) {
    // ...code that handles underflow...
    return -1;
  } else if (rc == Number::DivideByZero) {
    // ...code that handles divide-by-zero...
    return -1;
  }
}
\end{lstlisting}
\end{frame}

\subsection{Исключения в конструкторах}
\begin{frame}[fragile]{Исключения в конструкторах}
\begin{enumerate}
    \item У конструктора нет возвращаемого значения.
    \item Практически единственный способ сообщить об ошибке - кинуть исключение.
    \item Память будет очищена, для членов класса - вызван деструктор.
    \item Для самого объекта деструктор вызван не будет.
    \item Члены класса должны сами за собой подчищать (RAII).
\end{enumerate}
\end{frame}

\subsection{Исключения в деструкторах}
\begin{frame}[fragile]{Исключения в деструкторах}
\begin{enumerate}
    \item Исключения из деструктора практически никогда нельзя кидать. Почему?
    \item Если в программе возникает исключение во время обработки другого исключения, происходит terminate().
    \item При раскрутке стека вызываются деструкторы всех объектов на данном уровне стека.
    \item Вывод?
    \item Если используете какие-то функции внутри деструктора, и они кидают исключения - обрабатывайте исклчюения внутри деструктора.
\end{enumerate}
\end{frame}

\subsection{Как кидать и ловить?}
\begin{frame}[fragile]{Как кидать и ловить?}
\begin{enumerate}
    \item Исключение может иметь любой тип.
    \item Обычно тип исключения наследуют от std::exception или производных.
    \item Ловить можно по: значению, ссылке, указателю.
    \item Наиболее правильно - по ссылке.
\end{enumerate}
\end{frame}

\begin{frame}[fragile]{Почему не по указателю?}
\begin{lstlisting}[language=C++]
MyException x;
void f()
{
  MyException y;
  try {
    switch ((rand() >> 8) % 3) {
      case 0: throw new MyException;
      case 1: throw &x;
      case 2: throw &y;
    }
  }
  catch (MyException* p) {
    // Здесь нам удалять p или нет?
  }
}
\end{lstlisting}
\end{frame}

\subsection{Что такое throw; и catch (...)?}
\begin{frame}[fragile]{Что такое throw; и catch (...)?}
\begin{lstlisting}[language=C++]
void handleException()
{
  try {
    throw; // пробросить текущее исключение дальше
  }
  catch (MyException& e) {
    // обработка MyException
  }
  catch (YourException& e) {
    // обработка YourException
  }
}
void f()
{
  try {
    // что-то, что кидает исключения
  }
  catch (...) { // ловит все исключения
    handleException();
  }
}
\end{lstlisting}
\end{frame}


\subsection{Исключения и полиморфизм}
\begin{frame}[fragile]{Исключения и полиморфизм}
\begin{lstlisting}[language=C++]
class MyExceptionBase { };
class MyExceptionDerived : public MyExceptionBase { };
void f(MyExceptionBase& e)
{
  // ...
  throw e;
}
void g()
{
  MyExceptionDerived e;
  try {
    f(e);
  }
  catch (MyExceptionDerived& e) {
    // обработка MyExceptionDerived
  }
  catch (...) {
    // обработка остальных исключений
    // поток выполнения пойдёт сюда
  }
}
\end{lstlisting}
\end{frame}

\subsection{Спецификации исклчючений}
\begin{frame}[fragile]{Спецификации исклчючений}
\begin{lstlisting}[language=C++]
void f() throw(int); // OK: function declaration
void (*fp)() throw (int); // OK: pointer to function declaration
void g(void pfa() throw(int)); // OK: pointer to function parameter declaration
typedef int (*pf)() throw(int); // Error: typedef declaration

void f() throw() // не бросает исключений
void f() throw(...) // может бросать исклчючения
\end{lstlisting}
Если функция кидает исключение, которое имеет тип, не указанный в спецификации, то вызывается std::unexpected. По умолчанию она вызывает std::terminate, но может быть заменена на пользовательскую функцию. \\
Спецификация исключений в таком виде - довольно бесполезная и дорогая штука, планируют убрать из языка.
\end{frame}

\subsection{SEH}
\begin{frame}[fragile]{SEH}
\begin{enumerate}
    \item Windows-only
    \item Позволяют ловить исключения процессора: access violation, illegal instruction, divide by zero.
    \item Механизм, отличный от исключений C++
    \item Один компилятор. Одна ОС. Не «чистый С++». 
\end{enumerate}
\begin{lstlisting}[language=C++]
__try 
{
   // guarded code
}
__except ( expression )
{
   // exception handler code
}
\end{lstlisting}
\end{frame}

\appendix
\section<presentation>*{\appendixname}
\subsection<presentation>*{Useful links}

\begin{frame}[allowframebreaks]
  \frametitle<presentation>{Полезные ссылки}
    
  \begin{thebibliography}{10}
{
  \beamertemplatebookbibitems
  % Start with overview books.

  \bibitem{faq}
  \texttt{C++ FAQ: Exceptions}
  \newblock \href{https://isocpp.org/wiki/faq/exceptions}{\texttt{https://isocpp.org/wiki/faq/exceptions}}
  
  \bibitem{habr1}
  \texttt{Habr: Коды возврата vs исключения — битва за контроль ошибок}
  \newblock \href{https://habrahabr.ru/post/130611/}{\texttt{https://habrahabr.ru/post/130611/}}
  
  \bibitem{habr2}
  \texttt{Habr: Коды возврата & исключения}
  \newblock \href{https://habrahabr.ru/post/131212/}{\texttt{https://habrahabr.ru/post/131212/}}
  
  \bibitem{habr3}
  \texttt{Habr: Обработка Segmentation Fault в C++ (про SEH)}
  \newblock \href{https://habrahabr.ru/post/131412/}{\texttt{https://habrahabr.ru/post/131412/}}
  
  \bibitem{msdn}
  \texttt{MSDN: Обработка исключений С++}
  \newblock \href{https://msdn.microsoft.com/ru-ru/library/4t3saedz.aspx}{\texttt{https://msdn.microsoft.com/ru-ru/library/4t3saedz.aspx}}
  
   \bibitem{cppref1}
  \texttt{CppReference: dynamic exception specification}
  \newblock \href{http://en.cppreference.com/w/cpp/language/except_spec}{\texttt{ http://en.cppreference.com/w/cpp/language/except_spec}}
}
    
  \end{thebibliography}
\end{frame}

\end{document}


