 4\documentclass[10pt]{beamer}

\usetheme{CambridgeUS}
\usepackage[english, russian]{babel}
\usepackage[utf8]{inputenc}
\usepackage{caption}
\usepackage{etoolbox}
\usepackage{multicol}
\usepackage{listings}
\usepackage{color}

\definecolor{mygreen}{rgb}{0,0.6,0}
\lstset{
  basicstyle=\ttfamily\footnotesize,        % the size of the fonts that are used for the code
  breaklines=true,                 % automatic line breaking only at whitespace
  captionpos=b,                    % sets the caption-position to bottom
  commentstyle=\color{mygreen},    % comment style
  keywordstyle=\color{blue},       % keyword style
  stringstyle=\color{red},     % string literal style
  showstringspaces=false,
  morekeywords={include, printf},
  texcl=true     %<---- added
}

\title[\href{https://goo.gl/NRgp8K}{https://goo.gl/NRgp8K} (Term 3)]{Краткое описание стандарта C\texttt{++}11}
\author[Гусев Илья]{Гусев Илья}
\institute[МФТИ] 
{Московский физико-технический институт\\*}
\date{Москва, 2017}
\subject{Computer Science}

\begin{document}

\begin{frame}
  \titlepage
\end{frame}

\begin{frame}{Содержание}
\begin{multicols}{2}
\tableofcontents
\end{multicols}
\end{frame}

\begin{frame}{Список фич}
\begin{itemize}
\item{Упрощение:\\ \texttt{\textcolor{blue}{\hyperlink{Auto}{Auto}, \hyperlink{Array}{array}, \hyperlink{Tuple}{tuple},} forward\_list, \textcolor{red}{unordered containers}, \textcolor{blue}{\hyperlink{Range-for}{range-for statement}, \hyperlink{Regex}{regex}, \hyperlink{Nullptr}{nullptr}, \hyperlink{In-class member initializers}{in-class member initializers}}, raw string literals }}
\item{Обобщение:\\ \texttt{\textcolor{blue}{\hyperlink{Uniform initialization}{Uniform initialization}, \hyperlink{Inherited constructors}{inherited constructors}, \hyperlink{Delegating constructors}{delegating constructors}, \hyperlink{Static assertions}{static assertions}}, \textcolor{green}{threads}, template alias, \textcolor{red}{variadic templates}}}
\item{Расширение:\\ \texttt{\textcolor{blue}{\hyperlink{Lambdas}{Lambdas}, \hyperlink{Default and delete}{default and delete}, \hyperlink{Suffix return type}{suffix return type}, \hyperlink{Smart pointers}{smart pointers}, \hyperlink{Override-final}{override-final}}, enum class, constexpr, noexcept, user-defined literals, alignment, \textcolor{red}{rvalue}, \textcolor{green}{TLS, async, future-promise}}}
\item{Исправление:\\ \texttt{\textcolor{blue}{Right-angle brackets}, unions, PODs}}
\end{itemize}
\end{frame}

\section{Упрощение}

\subsection{Auto}
\hypertarget{Auto}{}
\begin{frame}[fragile]{Auto}{Вывод типа в инициализаторе}
Компилятор в состоянии предугадать тип пременной при её инициализации. Ключевое слово - auto. Пример:
\begin{lstlisting}[language=C++]
auto i = 7;
\end{lstlisting}
Компилятор понимает, что i - переменная типа int. Более сложный пример:
\begin{lstlisting}[language=C++]
template<class T, class U> void multiply(
    const vector<T>& vt, 
    const vector<U>& vu)
{
    auto tmp = vt[0] * vu[0];
}
\end{lstlisting}
\end{frame}


\begin{frame}[fragile]{Auto}{Вывод типа в инициализаторе}
Было:
\begin{lstlisting}[language=C++]
template<class T> void printall( const vector<T>& v )
{
    for( typename vector<T>::const_iterator p = v.begin(); 
        p!=v.end(); ++p ) 
    {
        cout << *p << "\n";
    }
}
\end{lstlisting}
\vspace{5mm}
Стало:
\begin{lstlisting}[language=C++]
template<class T> void printall( const vector<T>& v )
{
    for( auto p = v.begin(); p!=v.end(); ++p ) {
        cout << *p << "\n";
    }
}
\end{lstlisting}
\end{frame}

\begin{frame}[fragile]{Auto}{Вывод типа в инициализаторе}
Здесь же нужно упомянуть decltype. То, что раньше называлось typeof.
\begin{lstlisting}[language=C++]

void f( const vector<int>& a, vector<float>& b )
{
    typedef decltype( a[0] * b[0] ) Tmp;
    for( int i = 0; i < b.size(); ++i ) {
        Tmp* p = new Tmp( a[i] * b[i] );
    }
}
\end{lstlisting}
\end{frame}

\subsection{Range-for}
\hypertarget{Range-for}{}
\begin{frame}[fragile]{Range-for statement}{Удобный цикл}
 Аналог foreach цикла в других языках. Можно ипользовать для большинства стандартных контейнеров и для всего, где определён begin() и end(), в том числе и для пользовательских классов.
\begin{lstlisting}[language=C++]
void f( vector<double>& v )
{
    for( auto x : v ) cout << x << '\n';
    // По ссылке можем менять значение.
    for( auto& x : v ) ++x;	
}
\end{lstlisting}
\end{frame}

\subsection{Array}
\hypertarget{Array}{}
\begin{frame}[fragile]{Array}{Новый простой стандартный контейнер}
 \begin{itemize}
 \item{Элемент STL, std::array}
 \item{Как обычный статический массив, но с поддержкой всех итераторных методов STL}
 \item{Поддержка initializer list}
 \item{Знает свой размер}
 \item{Не спутать с указателем}
  \end{itemize}
\end{frame}

\begin{frame}[fragile]{Array}{Новый простой стандартный контейнер}
\begin{lstlisting}[language=C++]
array<int, 6> a = { 1, 2, 3 };
a[3] = 4;

// x равен 0, потому что  по умолчанию элементы zero initialized.
int x = a[5]; 

// Error: std::array неявно не преобразовывается у указателю.
int* p1 = a; 
int* p2 = a.data(); // Ok: указатель на первый элемент.
array<int> a3 = { 1, 2, 3 }; // Error: не указан размер.
array<int, 0> a0; // Ok: пустой массив.
int* p = a0.data(); // Unspecified; не делайте так.
\end{lstlisting}
\end{frame}


\subsection{Tuple}
\hypertarget{Tuple}{}
\begin{frame}[fragile]{Tuple}{Кортеж в C\texttt{++}}
\begin{itemize}
\item{Элемент STL, std::tuple}
 \item{Удобен для хранения разнородных объектов}
 \item{Частный случай - std::pair}
\end{itemize}
\vspace{5mm}
\begin{lstlisting}[language=C++]
std::tuple<string,int> t2( "Word", 123 );
// У t будет тип tuple<string,int,double>.
auto t = make_tuple( string( "Letter" ), 10, 1.23 );	
string s = get<0>(t);
int x = get<1>(t);
double d = get<2>(t);
\end{lstlisting}
\end{frame}

\subsection{nullptr}
\hypertarget{Nullptr}{}
\begin{frame}[fragile]{nullptr}{Выделеныый нулевой указатель}
\begin{lstlisting}[language=C++]
char* p = nullptr;
int* q = nullptr;
char* p2 = 0; // 0 работает, p==p2.

void f(int);
void f(char*);
f(0); // Вызов f(int).
f(nullptr); // Вызов f(char*).

void g(int);
g(nullptr); // Error: nullptr не int.
int i = nullptr; // Error: nullptr не int.
\end{lstlisting}
\end{frame}

\subsection{In-class member initializers}
\hypertarget{In-class member initializers}{}
\begin{frame}[fragile]{In-class member initializers}{Инициализируем поля непосредственно в классе}
\begin{lstlisting}[language=C++]
class A {
public:
    int a = 7;
};
\end{lstlisting}
эквивалентно
\begin{lstlisting}[language=C++]
class A {
public:
    int a;
    A(): a(7) {}
};
\end{lstlisting}
Если есть и то, и то - constructor initialization > in-class initialization.
\end{frame}

\subsection{Regex}
\hypertarget{Regex}{}
\begin{frame}[fragile]{Regex}{Регулярные выражения в C\texttt{++}}
Семейство функций \texttt{regex\_replace, regex\_search, \\regex\_match}.

\begin{lstlisting}[language=C++]
std::string text = "Quick brown fox";
std::regex vowel_re( "a|o|e|u|i" );
std::regex_replace( text, vowel_re, "[X]" );
// Q[X][X]ck br[X]wn f[X]x
\end{lstlisting}
\end{frame}

\section{Обобщение}
\subsection{Uniform initialization}
\hypertarget{Uniform initialization}{}
\begin{frame}[fragile]{Uniform initialization}{Одинаковая инцицализация в разных ситуациях}
Новый тип - \texttt{std::initializer\_list<T>}. Любая длина списка, но один и тот же тип элементов.
\begin{lstlisting}[language=C++]
void f( initializer_list<int> );
f( {1, 2} );
f( {23, 345, 4567, 56789} );
f({}); // Пустой список инициализации.
f{ 1,2 }; // Error: нет () при вызове функцииы
years.insert( { {"Bjarne", "Stroustrup"}, {1950, 1975, 1985} } );
\end{lstlisting}
\end{frame}

\begin{frame}[fragile]{Uniform initialization}{Одинаковая инцицализация в разных ситуациях}
Раньше:
\begin{lstlisting}[language=C++]
// Ok: можно использовать для стат. массивов
string a[] = { "foo", " bar" }; 
// Error: для std::vector уже нельзя
vector<string> v = { "foo", " bar" }; 
void f( string a[] );
f( { "foo", " bar" } ); // Syntax error: block as argument

int a = 2; // Обычное присваивание
int aa[] = { 2, 3 }; // Обычное присваивание массиву
complex z( 1, 2 ); // Инициализация в функциональном стиле
// Инициализация в функциональном стиле с преобразованием типа
x = Ptr(y); 

int a(1); // Определение перременной.
int b(); // Объявление переменной.
int b(foo); // Определение перременной или объявление функции.
\end{lstlisting}
\end{frame}

\begin{frame}[fragile]{Uniform initialization}{Одинаковая инцицализация в разных ситуациях}
Теперь можно так:
\begin{lstlisting}[language=C++]
X x1 = X{ 1, 2 }; 
X x2 = { 1, 2 };
X x3{ 1, 2 }; 
X* p = new X{1,2}; 

struct D : X {
    D( int x, int y ) :X{ x, y } {};
};
struct S {
    int a[3];
    S( int x, int y, int z ) :a{ x, y, z } {};
};

X x{a}; 
X* p = new X{a};
z = X{a};// Переобразование типа.
f( {a} );// Аргумент функции (типа X).
return {a}; // Возвращаемое значение функции (типа X)
\end{lstlisting}
\end{frame}

\subsection{Inherited constructors}
\hypertarget{Inherited constructors}{}
\begin{frame}[fragile]{Inherited constructors}{Конструкторы базового класса в области видимости наследника}
Была проблема:
\begin{lstlisting}[language=C++]

struct B {
    void f(double);
};
struct D : B {
    void f(int);
};
B b; b.f(4.5);	// Ок.
// Неочивидная особенность: вызов f(int) с аргументом 4.
D d; d.f(4.5);	
\end{lstlisting}
\end{frame}
\begin{frame}[fragile]{Inherited constructors}{Конструкторы базового класса в области видимости наследника}
И было решение:
\begin{lstlisting}[language=C++]
struct B {
    void f(double);
};
struct D : B {
    // Вносим все f() базового класса в область видимости.
    using B::f; 
    void f(int);
};

B b;   b.f(4.5);	// Ок.
// Ок: вызов D::f(double), которая на самом деле B::f(double).
D d;   d.f(4.5);	
\end{lstlisting}
\end{frame}
\begin{frame}[fragile]{Inherited constructors}{Конструкторы базового класса в области видимости наследника}
Раньше подобный трюк не работал с конструкторами, теперь работает.
\begin{lstlisting}[language=C++]
class Derived : public Base { 
public: 
    using Base::f;    // Работает в C++98.
    void f(char);
    
    using Base::Base; // Работает только с C++11.
    Derived(char);
}; 
\end{lstlisting}
\end{frame}

\subsection{Delegating constructors}
\hypertarget{Delegating constructors}{}
\begin{frame}[fragile]{Delegating constructors}{Вызов одного конструктора в другом}
Раньше нужно было извернутсья, чтобы 2 конструктора разделяли какую-то функциональность:
\begin{lstlisting}[language=C++]
class X {
    int a;
    void init(int x) { /* ... */ }
public:
    X(int x) { init(x); }
    X() { init(42); }
};
\end{lstlisting}
Теперь всё просто:
\begin{lstlisting}[language=C++]
class X {
    int a;
public:
    X(int x) { { /* ... */ }
    X(): X{42} { }
};
\end{lstlisting}
\end{frame}

\subsection{Static assertions}
\hypertarget{Static assertions}{}
\begin{frame}[fragile]{Static assertions}{Проверки на этапе компиляции}
Суть ясна, но не сразу понятно, зачем это нужно. Варианты использования - провекра платформы при компиляции и контроль целостности enum'ов.
\begin{lstlisting}[language=C++]
static_assert( sizeof(long) >= 8, 
    "64-bit code generation required for this library." );
struct S { X m1; Y m2; };
static_assert( sizeof(S) == sizeof(X) + sizeof(Y), 
    "unexpected padding in S" );

int f(int* p, int n)
{
    // Error: не работает для runtime-проверок.
    static_assert( p == 0, "p is not null" ); 
}

\end{lstlisting}
\end{frame}

\section{Расширение}
\subsection{Lambdas}
\hypertarget{Lambdas}{}
\begin{frame}[fragile]{Lambdas}{Лямбда-функции}
Безымянные функции, которые могут захватывать пременные из внешней по отношению к ним области видимости.
\begin{lstlisting}[language=C++]
vector<int> v = {50, -10, 20, -30};
std::sort(v.begin(), v.end());
// v: { -30, -10, 20, 50 }
std::sort(v.begin(), v.end(), 
    [](int a, int b) { return abs(a)<abs(b); });
// v: { -10, 20, -30, 50 }
\end{lstlisting}
\end{frame}
\begin{frame}[fragile]{Lambdas}{Лямбда-функции. Захват переменных}
\texttt{[]} - нет захвата переменных.\\\texttt{[\&]} - захват всех переменных по ссылке.\\ \texttt{[=]} - захват всех переменных по значению.\\ \texttt{[\&a, \&b]} - захват a и b  по сслыке.\\ \texttt{[a, b]} - захват a и b по значению.\\\texttt{[in, \&out]} - захват in по значению, а out — по ссылке\\ \texttt{[=, \&out1, \&out2]} - захват всех переменных по значению, кроме out1 и out2, которые захватываются по ссылке.\\\texttt{[\&, x, \&y]} - захват всех переменных по ссылке, кроме x.
\begin{lstlisting}[language=C++]
void f( vector<Record>& v )
{
    vector<int> indices( v.size() );
    int count = 0;
    generate( indices.begin(), indices.end(), 
        [&count](){ return count++; } );
    std::sort( indices.begin(), indices.end(), 
        [&](int a, int b) { return v[a].name<v[b].name; } );
}
\end{lstlisting}
\end{frame}
\begin{frame}[fragile]{Lambdas}{Лямбда-функции. Тип возвращаемого значения}
По умолчанию лямбда возвращает void. При наличии одного return, компилятор вычисляет тип возвращаемого значения. Если же в лямбде присутствует ветвление, то на компилятор полагаться уже нельзя:
\begin{lstlisting}[language=C++]
[] (int n) 
{ 
    if (n % 2 == 0)
        return n / 2.0;
    else
        return n * n;
});
\end{lstlisting}
Компилятор не может самостоятельно вычислить тип возвращаемого значения, поэтому мы должны его указать явно:
\begin{lstlisting}[language=C++]
[] (int n) -> double // suffix return type
{ 
    if (n % 2 == 0)
        return n / 2.0;
    else
        return n * n;
});
\end{lstlisting}
\end{frame}

\subsection{Default and delete}
\hypertarget{Default and delete}{}
\begin{frame}[fragile]{Default and delete}{Контроль поведения по умолчанию}
Мы можем запрещать использование функций, а также заставлять использовать версии по умолчанию.
\begin{lstlisting}[language=C++]
class X {
public:
    X& operator=(const X&) = delete; // Запретили копирование.
    X(const X&) = delete;
};
\end{lstlisting}
\begin{lstlisting}[language=C++]
class Y {
public:
    // Явно заставили использовать копирование по умолчанию.
    Y& operator=(const Y&) = default; 
    Y(const Y&) = default;
};
\end{lstlisting}
\begin{lstlisting}[language=C++]
struct Z {
    Z(long long); // Может инициализироваться с long long,
    Z(long) = delete; // но ни с чем больше.
};
\end{lstlisting}
\end{frame}

\subsection{Smart pointers}
\hypertarget{Smart pointers}{}
\begin{frame}[fragile]{Smart pointers}{Умные указатели - владение}
\texttt{Unique\_ptr} - контейнер с семантикой владения. 
\begin{enumerate}
    \item Владеет объектом, на который указывает.
    \item Запрещает копирование, но разрешает перемещение (move). 
    \item При уничтожении указателя, уничтожается и объект, которым он владеет. 
    \item Позволяет возвращать из функции объекты, выделенные на динамической памяти.
    \item Кроме того корректно уничтожает такие объекты при раскрутке стека во время проталкивания исключений.
    \item Опирается на rvalue семантику.
\end{enumerate}
\end{frame}
\begin{frame}[fragile]{Smart pointers}{Умные указатели - владение}
Примеры:
\begin{lstlisting}[language=C++]
unique_ptr<X> f()
{
    unique_ptr<X> p( new X );
    // Можно кидать исключения, всё будет ок.
    return p; // передали право владения наружу f()
}
void g()
{
    // Перемещаем с помощью move constructor.
    unique_ptr<X> q = f(); 
    q->memfct(2); // Используем q.
    X x = *q; // Можео скопировать сам объект.
} // q и объект, которым он владеет уничтожается при выходе.
\end{lstlisting}
\end{frame}
\begin{frame}[fragile]{Smart pointers}{Умные указатели - разделяемое владение}
\begin{enumerate}
\item \texttt{Shared\_ptr} - контейнер с семантикой раздеяемого владения. 
\item \texttt{Weak\_ptr} - контейнер, дополняющий \texttt{shared\_ptr}, нужен, чтобы не было циклических зависимостей и, соответственно, неумирающих объектов.\\ 
\item \texttt{Shared\_ptr} реализован через счётчик ссылок, каждый \texttt{shared\_ptr}, указывающий на объект даёт \texttt{+1} к этому счётчику. 
\item Если счётчик становится равен нулю, объект уничтожается. 
\item \texttt{Weak\_ptr} не меняет счётчик ссылок.
\item При неправильном использовании возможно возникновение циклических зависимостей.
\item \texttt{Shared\_ptr} дорого стоит, особенно в параллельной среде.
\end{enumerate}
\end{frame}
\begin{frame}[fragile]{Smart pointers}{Умные указатели - разделяемое владение}
Иллюстрация к счётчику ссылок:
\begin{lstlisting}[language=C++]
void test()
{
    shared_ptr<int> p1(new int); // Счётчик равен 1.
    {
        shared_ptr<int> p2(p1); // Счётчик равен 2.
        {
            shared_ptr<int> p3(p1); // Счётчик равен 3.
        } // Счётчик равен 2.
    } // Счётчик равен 1.
} // Счётчик равен 0, int уничтожается.

\end{lstlisting}
\end{frame}

\subsection{Suffix return type}
\hypertarget{Suffix return type}{}
\begin{frame}[fragile]{Suffix return type}{Помощь в выводе типа возвращаемого значения}
Иногда мы должны помочь компилятору вывести тип возвращаемого значения, без нас он этого сделать не может. Мы не можем написать \texttt{decltype( x * y )} где обычно (вместо \texttt{auto}), так как \texttt{x} и \texttt{y} там ещё не определены. Таким образом, эта штука нужна в основном именно для обхода области видимости. Используется в лямбдах.
\begin{lstlisting}[language=C++]
template<class T, class U>
auto mul( T x, U y ) -> decltype( x * y )
{
    return x * y;
}
\end{lstlisting}
\end{frame}

\subsection{Override-final}
\hypertarget{Override-final}{}
\begin{frame}[fragile]{Override-final}{Явные виртуальные функции - override}
\begin{lstlisting}[language=C++]
struct B {
    virtual void f();
    virtual void g() const;
    virtual void h(char);
    void k(); // Не вирутальная.
};
struct D : B {
    void f(); // Перезаписывает B::f().
    void g();// Не перезаписывает B::g() (не та сигнатура).
    virtual void h(char); // Перезаписывает B::h().
    void k(); // Не перезаписывает B::k() (B::k() не виртуальна).
};
// Сейчас можно так
struct D : B {
    void f() override; // OK: overrides B::f().
    void g() override; // Error: не та сигнатура.
    virtual void h(char); // Перезаписывает B::h(), warning.
    void k() override; // Error: B::k() не виртуальна.
};
\end{lstlisting}
\end{frame}
\begin{frame}[fragile]{Override-final}{Явные виртуальные функции - final}
\begin{lstlisting}[language=C++]
struct B {
    // Говорим, что перезаписывать нельзя.
    virtual void f() const final; 
    virtual void g();
};

struct D : B {
    // Error: D::f попытка перезаписи финальной B::f.
    void f() const; 
    void g(); // OK.
};
\end{lstlisting}
\end{frame}



\section{Исправление}

\subsection{Right-angle brackets}
\begin{frame}[fragile]{Right-angle brackets}{Пробел не нужен!}
\begin{lstlisting}[language=C++]
list<vector<string>> lvs;
\end{lstlisting}
Раньше была синтаксическая ошибка, теперь нет.
\end{frame}

\appendix
\section<presentation>*{\appendixname}
\subsection<presentation>*{Useful links}

\begin{frame}[allowframebreaks]
  \frametitle<presentation>{Полезные ссылки}
    
  \begin{thebibliography}{10}
{
  \beamertemplatebookbibitems
  % Start with overview books.


  \bibitem{C++11 - the new ISO C++ standard}
  \texttt{C++11 - the new ISO C++ standard}
  \newblock Bjarne Stroustrup
  \newblock \href{http://stroustrup.com/C++11FAQ.html}{\texttt{http://www.stroustrup.com/C++11FAQ.html}}
  
  \bibitem{C++ Super-FAQ}
   \texttt{C++ Super-FAQ}
   \newblock Bjarne Stroustrup, Marshall Cline
  \newblock \href{https://isocpp.org/faq}{\texttt{https://isocpp.org/faq}}
  
  \bibitem{C++11 draft}
  \texttt{N3337 2012 Draft (C++11)}
  \newblock ISO/IEC
  \newblock \href{http://open-std.org/jtc1/sc22/wg21/docs/papers/2012/n3337.pdf}{\texttt{http://open-std.org/jtc1/sc22/wg21/docs/papers/2012/n3337.pdf}}
  
  \bibitem{Habr lambda}
  \texttt{C++0x (С++11). Лямбда-выражения }
  \newblock Сергей Олендаренко 
  \newblock \href{https://habrahabr.ru/post/66021/}{\texttt{https://habrahabr.ru/post/66021/}}
  
  
}
    
  \end{thebibliography}
\end{frame}

\end{document}


